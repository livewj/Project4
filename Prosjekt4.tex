% standard LaTeX packages
\documentclass[11pt,a4paper]{article}

\usepackage[utf8]{inputenc}
\usepackage[T1]{fontenc}
\usepackage[english]{babel}
\usepackage[margin=3cm]{geometry}
\usepackage{parskip}
\usepackage{xifthen}
\usepackage{placeins} %Floatbarrier
\usepackage{graphicx} %package to manage images
\usepackage{subcaption}

% math packages
\usepackage{mathtools,amsfonts,amssymb,mathdots}
\usepackage{siunitx}

\usepackage[rightcaption]{sidecap}

\newcommand{\dd}[1]{\mathrm{d}#1} %numbering
% plotting and tables
\usepackage{tikz}
\usepackage{pgfplots}
\usepackage{pgfplotstable}
\usepackage{caption}
% other packages
\usepackage{filecontents}


\begin{document}

\title{Prosjekt 4: Faseoverganger i magnetiske systemer }
\author{Live Wang Jensen}
\date{\today}

\maketitle

\begin{abstract}
Målet med dette prosjektet er å implementere Ising modellen for et 2-dimensjonal kvadratisk gitter. Denne modellen beskriver et magnetisk materiale, og vi har forenklet modellen ved å bruke et system som består av partikler, som vi antar er festet i et gitter. Disse partiklene kan enten ha spinn opp eller spinn ned, og kan vekselvirke med hverandre ved å endre spinntilstand. Vi har brukt Metropolis algoritmen for å beregne sannsynlighetstettheten til de ulike tilstandene (?) under en Monte Carlo simulering. Vi har sett på de termodynamiske parameterene, som energi $E$, magnetisk moment $M$, spesifikk varmekapasitet $C_V$ og susceptibiliteten $\chi$ som funksjon av temperatur $T$ ved den kritiske temperaturen $T_C$. KONKRETE TALL

\end{abstract}

\tableofcontents

\clearpage
\section{Introduksjon}
HVA HAR VI GJORT
Målet med dette prosjektet er å studere faseoverganger i to dimensjoner ved hjelp av den populære Ising modellen. Ved en gitt kritisk temperatur vil denne modellen vise at vi får en faseovergang fra en magnetisk fase til en fase med null magnetisering.
bla bla



\section{Teori}
Ising modellen gjør oss i stand til å beskrive oppførselen til et magnetisk materiale som funksjon av termisk energi og ytre magnetfelt. Vi antar her at gitteret vårt er kvadratisk og består av et rutenett av atomer, som enten kan ha spinn opp (+1) eller ned (-1). Energien i Ising modellen kan i sin enkleste form beskrives ved 
\begin{equation}
E = -J\sum_{<kl>}^N s_ks_l - \mathcal{ B} \sum_k^N s_k
\end{equation}
hvor $s_k = \pm 1$. Størrelsen $N$ representerer det totale antlall spinn vi kan ha, mens $J$ er koblingskonstanten som forteller oss noe om styrken på vekselvirkningen mellom to nabospinn. Symbolet $<kl>$ indikerer at vi kun skal summere over de nærmeste naboene. I vårt tilfelle har vi ikke noe ytre magnetfelt $\mathcal{B}$, slik at det siste leddet forsvinner. Vi antar at vi har en ferromagnetisk struktur, slik at $J >$ 0. I vårt tilfelle bruker vi $J$ = 1. Vi kommer til å bruke periodiske grensebetingelser og \textbf{Metropolisalgoritmen} til å modellere vårt system. \\

For å kunne beregne forventningsverdier som for eksempel forventningsverdien til energien $\langle E \rangle$ eller magnetiseringen $\langle M \rangle$ i statistisk fysikk, trenger vi sannsynlighetsfordelingen, som forteller oss om sannsynligheten for å finne systemet vårt i en gitt konfigurasjon $i$:

\begin{equation}
P_i(\beta) = \frac{e^{-\beta E_i}}{Z}
\end{equation}

hvor $\beta = 1/kT$, $k$ er Boltzmanns konstant og $E_i$ er energien til tilstand nummer $i$. $Z$ er partisjonsfunksjonen for et kanonisk ensemble. Det betyr at ($N,V,T$) (antall spinn, volum og temperatur) holdes konstant. $Z$ er da gitt ved
\begin{equation}
Z = \sum_{i=1}^M e^{-\beta E_i }
\end{equation}
Her summerer vi over alle mulige mikrotilstander $M$. Energien til hver enkelt konfigurasjon $i$ er gitt ved
\begin{equation}
E_i = -J\sum_{<kl>}^N s_k s_l
\end{equation}

En gitt konfigurasjon i en endimensjonal Ising Modell hvor $\mathcal{B}$ = 0 med $N$ spinn kan for eksempel se slik ut

\[\uparrow_1 \quad \uparrow_2 \quad  ... \quad \uparrow_{i-1}  \quad \downarrow_i  \quad \downarrow_{i+1} \quad ...\quad \uparrow_{N-1} \quad \downarrow_N \]

hvor pilene symboliserer om spinnet peker opp (+1) eller ned (-1). Det finnes flere måter å håndtere grensebetingelsene på. Vi skal bruke \textbf{periodiske grensebetingelser}. Det betyr at nabospinnet til høyre for spinn $s_N$, skal ha samme verdi som spinn $s_1$. På samme måte skal naboen til venstre for $s_1$ ha samme verdi som $s_N$. \\

Det \textbf{magnetiske momentet} i et slik system er definert ved 
\begin{equation}
M = \sum_{j=1} ^N s_j
\end{equation}
hvor vi summer over alle spinn ved en gitt konfigurasjon $i$. I det endimensjonale tilfellet vil hvert spinn kun ha to naboer; en til høyre og en til venstre. Tar vi dette over til det todimensjonale plan, får hvert spinn plutselig fire naboer. Dersom vi ser på et 2$\times$2-gitter med periodiske grensebetingelser, ender vi opp med totalt $2^4$ = 16 ulike tilstander. Noen eksempler er gitt nedenfor:

\[ \uparrow \uparrow \quad \quad \uparrow \uparrow \quad \quad \downarrow \downarrow \quad \quad \downarrow \downarrow \]

\[ \uparrow \uparrow \quad \quad \uparrow \downarrow \quad \quad \uparrow \downarrow \quad \quad \downarrow \downarrow \]

\[ E = -8J \quad \quad E = 0 \quad \quad E = 0 \quad \quad E = -8J \]

Vi ser fort at det er flere ulike tilstander som har samme totalenergi. Antall tilstander som har samme totalenergi kalles for \textit{degenerasjonsgraden} til energien. Tabell \ref{degen}  gir en oversikt over de ulike tilstandenes energi, magnetisk moment og degerasjonsgrad for 2$\times$2 tilfellet.

\FloatBarrier
\begin{table}[!ht]
\centering
\caption{Energi og magnetisk moment for den todimensjonale Ising modellen, med N = 2$\times$2 spinn og periodiske grensebetingeler.}
\label{degen}
\begin{tabular}{llll}
\#Spinn opp & Degenerasjonsgrad & Energi & Magnetisk moment \\
4           & 1                 & -8J    & 4             \\
3           & 4                 & 0      & 2             \\
2           & 4                 & 0      & 0             \\
2           & 2                 & 8J     & 0             \\
1           & 4                 & 0      & -2            \\
0           & 1                 & -8J    & -4           
\end{tabular}
\end{table}
\FloatBarrier

Videre har vi at forventningsverdien til energien er 
\begin{equation}
\langle E \rangle = - \frac{\partial lnZ}{\partial \beta} 
\end{equation}

mens forventningsverdien til \textit{absoluttverdien} til det magnetiske momentet blir
\begin{equation}
\langle |M| \rangle = \frac{1}{Z} \sum_i^M |M_i| e^{-\beta E_i}
\end{equation}
hvor $M_i$ er det magnetiske momentet ved tilstanden $i$.


Den \textbf{spesifikke varmekapasiteten} er gitt ved
\begin{equation}
C_V = \frac{1}{kT^2}\sigma_E^2 = \frac{1}{kT^2}(\langle E^2 \rangle - \langle E \rangle^2 )
\end{equation}

hvor 
\begin{equation}
\langle E^2 \rangle = \frac{1}{Z} \sum_i E_i^2 e^{- \beta E_i}
\end{equation}

\textbf{Susceptibiliteten} er gitt ved
\begin{equation}
\chi = \frac{1}{kT} \sigma_{|M|}^2 = \frac{1}{kT}( \langle |M|^2 \rangle - \langle |M| \rangle^2 )
\end{equation}
hvor $|M |$ er absoluttverdien av det magnetiske momentet $M$.

\subsection{Faseoverganger}
Nær den kritiske temperaturen $T_C$ kan vi beskrive mange av de termodynamiske størrelsene ved hjelp av en såkalt potenslov. Dette er en egenskap som innebærer at frekvensen av en størrelses verdi er eksponensiell avtagende med den gjeldende verdien. For Ising modellen kan forventningsverdien til det magnetiske momentet beskvrives ved

\[ \langle M(T) \rangle \sim (T - T_C)^\beta \]

hvor $\beta$=1/8 er en såkalt \textit{kritisk eksponent}. Tilsvarende kan vi beskrive varmekapasiteten ved

\[ C_V \sim |T_C - T|^\alpha \]

og susceptibiliteten ved

\[ \chi(T) \sim |T_C - T|^\gamma  \]

hvor $\alpha$ = 0 og $\gamma$ = 7/4. I tillegg har vi \textit{korrelasjonslengden} som forventes å være på samme størrelse som gitteravstanden når $T >> T_C$. Siden spinnene blir mer og mer korrelerte når $T$ nærmer seg $T_C$, vil korrelasjonslengden øke i nærheten av den kritiske temperaturen. Siden vi jobber på en datamaskin, vil vi alltid ha en endelig størrelse på gitteret vårt. Det er mulig å skalere slik at vi kan knytte oppførselen til  et endelig gitter med resultatene til et uendelig stort gitter. Den kritiske temperaturen skaleres da som

\[T_C(L) - T_C(L=\infty ) = aL^{-1/\mu}  \]

hvor $a$ er en konstant og $\mu$ er definert ved ligningen

\[\xi (T) \sim |T_C - T|^{-\mu} \]

Ved å sette $T$ = $T_C$ får vi at
\[ \langle M(T) \rangle \sim (T-T_C)^\beta \rightarrow L^{-\beta/\mu} \]
og
\[C_V(T) \sim |T_C-T|^{-\gamma} \rightarrow L^{\alpha/\mu} \]
og
\[ \chi (T) \sim |T_C - T|^{-\alpha} \rightarrow L^{\gamma / \mu } \]

\subsection{Metropolis algoritmen}
For å beregne disse termodynamiske størrelsene på en datamaskin, skal vi bruke Metropolis algoritmen på Ising modellen med periodiske grensebetingelser. Denne algoritmen ble lagt frem av Metropolis mfl. i 1953. Vi kan generere nye tilstander fra den forrige tilstanden ved å bruke en sannsynlighetsmodell som avhenger av energidifferansen mellom tilstandene. I vårt tilfelle er Monte Carlo samplings-funksjonen vår det sammen som sannsynligheten for at systemet befinner seg i en tilstand $s$ gitt ved 

\[P_s = \frac{e^{-\beta E_s}}{Z} \]
hvor $Z$ er partisjonsfunksjonen for vårt kanoniske ensemble

\[Z(\beta) = \sum_s e^{\beta E_s}  \]

Siden vi trenger en oversikt over \textit{alle} tilstandene systemet kan ha for å kunne beregne partisjonsfunksjonen, sier det seg selv at dette vil være en tung og tidkrevende prosess. En Ising modell med $N=L\times L$ antall spinn i to dimensjoner, vil ha $2^N$ ulike konfigurasjoner. I Metropolis algoritmen slipper vi heldigvis unna dette, siden vi kun ser på \textit{forholdet} mellom sannsynlighetene. Kort fortalt går Metropolis algoritmen ut på at vi

\begin{enumerate}
\item Starter med en initaltilstand med energi $E_1$. Denne startkonfigurasjonen av spinn kan for eksempel ha alle spinnene pekende oppover, eller helt tilfeldig.
\item Endre starttilstanden ved å flippe kun ett spinn, og beregn den nye energien $E_2$.
\item Dersom differansen $\Delta E = E_2 - E_1 \le 0 $, aksepterer vi den nye konfigurasjonen. Energien er nå lavere enn det den startet med å være. På denne måten beveger vi oss mot et energiminimum.
\item Dersom differansen $\Delta E > 0$, beregn $w = e^{\beta \Delta E} $, og sammenlign $w$ med et tilfeldig trukket tall, $r$.
\item Hvis $r \le w$ godtas den nye konfigurasjonen. Hvis ikke beholdes den gamle konfigurasjonen. 
\item Oppdater de termodynamiske verdiene du vil beregne.
\item Når du har gått gjennom hele gitteret har du gått gjennom én Monte Carlo syklus. En slik syklus regnes som én måling. For å få flere målinger må du gå gjennom flere slike sykluser.
\end{enumerate}


\section{Resultater}
\subsection{2x2-gitter}
\textbf{Analytiske løsninger}\\
Vi starter med å se på et enkelt todimensjonalt system, nemlig et 2$\times$ 2 gitter. Partisjonsfunksjonen for et 2$\times$2-gitter kan finnes ved ligning (3). Dersom vi bruker verdiene gitt i tabell \ref{degen}, ser vi at vi kun har tre ulike energier; to konfigurasjoner har E = -8J, tolv konfigurasjoner har E = 0 og to konfigurasjoner med E = 8J. Setter vi dette inn i ligningen får vi at 

\[ Z = \sum_{i=1}^{16} e^{-\beta E_i} = e^{-\beta(-8J)} + 2e^{-\beta(8J)} + e^{- \beta(-8J)} + 12e^{-\beta(0)} \]
\[ = 2e^{8\beta J} + 2e^{-8 \beta J} + 12  \]

Videre kan vi finne $\langle E \rangle$ ved å bruke ligning (6), slik at 

\[\langle E \rangle = - \frac{\partial}{\partial \beta} \left[ ln(2e^{8\beta J} + 2e^{-8 \beta J} + 12) \right] \]
\[ = - \frac{16Je^{8\beta J} - 16Je^{-8\beta J}}{2e^{8\beta J} + 2e^{-8 \beta J} + 12 } \]

Vi vet fra tabell \ref{degen} at vi har én konfigrasjon med $M$ = 4, én med $M$ = -4, fire med $M$ = 2, fire med $M$ = -2 og seks konfigurasjoner med $M$ = 0. Vi bruker ligning (7) til å finne forventningsverdien til systemets magnetiske moment $\langle |M| \rangle$. Dette gir oss

\[ \langle |M| \rangle  = \frac{1}{2e^{8\beta J} + 2e^{-8 \beta J} + 12} \cdot (4e^{-\beta(-8J)} + 4\cdot 2 \cdot e^0 + 4\cdot 2 \cdot e^0 + | - 4|e^{-\beta(-8J)} )\]
\[ = \frac{8e^{8\beta J} + 16}{2e^{8\beta J} + 2e^{-8 \beta J} + 12} \]

Den spesifikke varmekapasiteten er gitt ved (8) og vi ender opp med
\[C_V = \frac{1}{kT^2}\left(\langle E^2 \rangle - \langle E \rangle^2 \right) = \frac{1}{kT^2}\left[ \frac{2\cdot 64J^2 (e^{8\beta J} + e^{-8\beta J})}{2e^{8\beta J} + 2e^{-8 \beta J} + 12} - \left( - \frac{16Je^{8\beta J} - 16Je^{-8\beta J}}{2e^{8\beta J} + 2e^{-8 \beta J} + 12 } \right)^2 \right]  \]

Til slutt har vi susceptibiliteten gitt ved ligning (10)
\[ \chi = \frac{1}{kT} (\langle |M|^2 \rangle - \langle |M| \rangle ^2 ) = \frac{1}{kT} \left[ \frac{32e^{8\beta J} + 32}{2e^{8\beta J} + 2e^{-8 \beta J} + 12 } - \left( \frac{8e^{8\beta J} + 16}{2e^{8\beta J} + 2e^{-8 \beta J} + 12 } \right)^2 \right]  \]

\textbf{Numeriske løsninger}\\
Vi kan sammenligne de analytiske løsningene med de numeriske løsningene funnet ved hjelp av Metropolis algoritmen. I koden \texttt{IsingModel$\_$4a.cpp} har Ising modellen blitt implementert sammen med de analytiske uttrykkene, slik at vi enkelt kan beregne $\langle E \rangle$, $\langle |M| \rangle$, $C_V$ og $\chi$. I koden er det brukt periodiske grensebetingelser med $L$ = 2, $J$ = 1 og $T$ = 1.0 (med enheten $kT/J$). I tabell \ref{num_vs_an} ser vi resultatene fra en kjøring av programmet. Her er antall Monte Carlo sykluser satt til $10^7$. Med disse verdiene oppnådde vi en presisjon på ett desimal. Vi ser på to ulike initialkonfigurasjoner, en hvor alle spinn starter med å peke oppover, og en hvor hvert spinn peker tilfeldig opp eller ned.

\FloatBarrier
\begin{table}[!ht]
\centering
\caption{Numeriske vs analytiske verdier for ulike termodynamiske størrelser. Her starter vi systemet vårt med alle spinn pekende oppover og antall Monte Carlo sykluser er $10^7$.}
\label{num_vs_an_up}
\begin{tabular}{|l|l|l|}
\hline
        & Numerisk & Analytisk \\ \hline
$<E>$   & -7.9842  & -7.9839   \\ \hline
$<|M|>$ & 3.9948   & 3.9946    \\ \hline
$C_V$   & 0.1258   & 0.1283    \\ \hline
$\chi$  & 0.0157   & 0.0160    \\ \hline
\end{tabular}
\end{table}
\FloatBarrier

\FloatBarrier
\begin{table}[!ht]
\centering
\caption{Numerisk vs analytiske verdier for ulike termodynamiske størrelser. Her starter vi med et system hvor de ulike spinnene peker i tilfeldig retning. Antall Monte Carlo sykluser er $10^7$.}
\label{num_vs_an_rand}
\begin{tabular}{|l|l|l|}
\hline
        & Numerisk & Analytisk \\ \hline
$<E>$   & -7.9836  & -7.9839   \\ \hline
$<|M|>$ & 4.0148   & 3.9946    \\ \hline
$C_V$   & 0.1306   & 0.1283    \\ \hline
$\chi$  & 1.0107   & 0.0160    \\ \hline
\end{tabular}
\end{table}
\FloatBarrier



\section{Diskusjon}
I tabellene \ref{num_vs_an_up} og \ref{num_vs_an_rand} for 2$\times$2 tilfellet ser vi at de numerisk beregnede verdiene stemmer godt overens med de analytiske løsningene. 
$\chi$ var den verdien som krevde flest Monte Carlo sykluser. 



\subsection{Optimalisering av kode}
-O3 ved kompilering. 
Vektorisering ved -Rpass=loop-vectorize
parallellisering av koden.
Interpolasjon av grafenei python.

Dersom vi sammenligner de numeriske verdiene i tabell \ref{num_vs_an} og \ref{num_vs_an}, ser vi at når vi starter med et ordnet system hvor alle spinn peker opp, vil vi også komme nærmere de analytiske verdiene., jo lenger unna det analytiske svaret kommer vi. 

\section{Konklusjon}

\section{Vedlegg}
Alle koder og resultater som er brukt i rapporten finnes på Github-adressen: \\
https://github.com/livewj/Project4


\bibliography{Referanser}
\begin{thebibliography}{9}  
    
\bibitem{}
   M. Hjort-Jensen: Computational physics, lecture notes 2015. Fysisk institutt, UiO, 2016.

\bibitem{}
   Oppgavetekst: Project 4, Fysisk institutt, UiO, 29.10.16
   
 \bibitem{}
   Termodynamiske uttrykk: "Energy Fluctuations in the Canonical Ensemble", Nick Oulette. <https://www.sccs.swarthmore.edu/users/02/no/pdfs/fluct.pdf>, 14.11.16
  
  \bibitem{}
  Utgangspunktet for kodene \texttt{IsingModel.cpp} og \texttt{ParaIsingModel.cpp} er hentet fra kursets Github-side:
  <https://github.com/CompPhysics/ComputationalPhysics/tree/master/\\
  doc/Programs/ParallelizationMPI>, 14.11.16
  
\bibitem{}
  Deler av koden i \texttt{ParaIsingModel.cpp} er hentet fra \\ <https://github.com/henrisro/Project4$\_$v2/blob/master/Parallellization$\%$20attempt/IsingMPI.cpp>,
  14.11.16

\bibitem{}
 Tips og triks for optimalisering av kode:
 "How to optimize codes, from vectorization to parallelization", Morten Hjorth-Jensen, Fysisk Institutt, UiO, 2016.
 
\bibitem{}
  Potenslov:
 <https://no.wikipedia.org/wiki/Potenslov>, 14.11.16

   
\end{thebibliography}
\end{document}


